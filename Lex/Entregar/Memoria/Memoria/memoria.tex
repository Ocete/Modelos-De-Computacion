\documentclass{article}
\usepackage[utf8]{inputenc}
\usepackage[english]{babel} 
\usepackage{url}
\usepackage{skak}

% Meta
\title{Práctica de Lex: Notación algebraica}
\author{José Antonio Álvarez}
\date{\today}

\begin{document}

\maketitle

\section{Introdución}

En esta memoria explicaremos el problema planteado asi como diversas herramientas relacionadas con el mismo como són el funcionamiento de la notación algebraica utilizada en el ajedrez y el uso del paquete \textbf{skak}. Se darán por sentados algunos conocimientos básicos de ajedrez.

\section{El problema}

Para los aficionados al ajedrez que no están muy versados en el tema (como personalmente me considero) la notación utilizado por los ajedrecistas puede ser algo confusa. El objetivo de este trabajo es tanto acercar dicha notación a este tipo de público mediante el uso de ejemplos ilustrativos como presentar una herramienta sencilla para transformar partidas descrita en esta notación a archivos del tipo \emph{.tex}. \\

Debido a lo complicado que resulta la instalación del paquete \textbf{skak} que he utilizado, adjunto una serie de ejemplos de uso del programa para que no sea necesario instalarlo. He seleccionado este paquete a pesar de dicha complicación por la facilidad que presenta a la hora de utilizarlo.

\section{La notación algebraica}

Esta es la notación utilizada por convenio en todo el mundo desde 1997, aunque varía en el nombre que recibe cada pieza según el idioma. Estos son los símbolos utilizados para denotar las piezas en inglés y español:

\begin{table}[h]
	\centering
	\caption{Piezas en notación algebraica}
	\label{my-label}
	\begin{tabular}{lcc}
		\textbf{}    & \multicolumn{1}{l}{Español} & \multicolumn{1}{l}{Ingles} \\
		Peón         & P                           & P (Pawn)                   \\
		Torre        & T                           & R (Rock)                   \\
		Caballo      & C                           & N (K\textbf\{n\}ight)      \\
		Álfil        & A                           & B (Bishop)                 \\
		Dama / Reina & D                           & Q (Queen)                  \\
		Rey          & K                           & K (King)                  
	\end{tabular}
\end{table}

Esto tendrá especial repercusión de cara al paquete \textbf{skak}, que utiliza la notación en inglés.

Siguiendo la notación algebraica, un movimiento se denota por la pieza que se mueve seguido de la casilla a la que se mueve. Así pues \textbf{Cc3} ó \textbf{Nc3} significa mover el caballo a la posición c3:

\newgame
\mainline{1.Nc3} \\
\showboard

Cuando se mueve un peón se omite el nombre de la pieza (P):

\mainline{1...b5} \\
\showboard

Cuando se come una pieza en el movimiento se añade \textbf{x} entre la pieza y la casilla:

\mainline{2.Nxb5} \\
\showboard \\

Es obvio que de esta forma se puede dar ambigüedad. Por ejemplo, en el siguiente tablero, ¿qué movimiento sería \textbf{xe5}? (le toca al blanco):

\newgame
\mainline{1.d4 e5 2.f4 Be7} \\
\showboard

Ambos peones pueden moverse a dicha posición. En estos casos se añade al principio del movimiento un símbolo adicional para distinguir que pieza se mueve. Si pueden identificarse por la columna, se añade la letra correspondiente a la columna. Si no es así, se utiliza la fila:

\mainline{3.fxe5} \\
\showboard

Si en un movimiento se realiza da jaque se añade al final el símbolo $+$. De ser un jaque mate se añade o bien $++$ o bien $\#$:

\mainline{3...Bh4+} \\
\showboard \\

Por último, el enroque corto se denota por \textbf{O-O} y el largo por \textbf{O-O-O}. A partir del siguiente tablero:

\newgame
\mainline{1. e3 d6 2. Bd3 Be6 3. Nf3 Nc6 4. h3 Qd7} \\
\showboard

Un enroque corto sería:

\mainline{5. O-O} \\
\showboard

Y un enroque largo:

\mainline{5...O-O-O} \\
\showboard

Hay además una serie de signos para realizas comentarios sobre jugadas específicas, como son \textbf{!} (jugada buena), \textbf{?} (jugada mala) y \textbf{!?} (jugada interesante). No he tenido en cuenta estos símbolos en mi trabajo. Para más información consultar \url{https://es.wikipedia.org/wiki/Notaci%C3%B3n_algebraica}.

\section{Paquete skak}

El uso es muy sencillo e intuitivo. La instalación ya es otro cantar. Para inicializar un tablero de 0 basta con hacer $\backslash newgame$ y cada movimiento sigue la notación algebraica explicada previamente (en inglés) con una pequeña salvedad: hay que indicar el número de jugada. De esta forma $\backslash mainline\{1.e4 e5\}$ Indicaría dos movimientos seguidos, uno por parte del jugador blanco y otro por aprte del negro. El siguiente movimiento tendrá que empezar \textbf{obligatoriamente} por \textbf{2}: $\backslash mainline\{2.b3\}$. La siguiente jugada, sin embargo, no es la 3 pues el jugador negro aún no ha movido dos piezas. Se denotaría por $\backslash mainline\{2...b6\}$. Por último, cada vez que queramos mostrar el tablero por pantalla utilizaremos $\backslash showboard$. Estos 4 movimientos se harían de la siguiente forma:

\newgame
\mainline{1.e4 e5} \\
\showboard

\mainline{2.b3} \\
\showboard

\mainline{2...b6} \\
\showboard

\section{Ejemplos ilustrativos}

En primer lugar, para probar sobre la notación algebraica de forma cómoda aconsejo la siguiente página: \url{https://www.chess.com/analysis-board-editor} \\ Conforme realizamos movimientos en el tablero de forma gráfica se actualizarán a la derecha los dichos movimientos en notación algebraica. \\

Respecto a los ejemplos adjuntos, esta misma memoria es una sencilla introducción al uso del paquete \textbf{skak}, así como el llamado \textbf{Defensa Siliciana}. Este último está hecho a mano.

Por último, adjunto 3 ejemplos de uso del programa. He de aclarar que aunque no se tenga el paquete \textbf{skak} instalado si que se puede generar el archivo \emph{.tex}. 

\begin{enumerate}
	\item El primer ejemplo es la denominada Partida Inmortal (\url{https://es.wikipedia.org/wiki/Inmortal_(partida_de_ajedrez)}).
	
	\item El segundo ejemplo es la denominada Partida Siempreviva (\url{https://es.wikipedia.org/wiki/Siempreviva_(partida_de_ajedrez)})
	
	\item El tercer ejemplo es la denominada Partida Inmortal de Mackenzie-Mason ( \url{http://www.ajedrezdeataque.com/12%20Visor/Visor1/Mackenzie-Mason.htm})
\end{enumerate}


\section{Bibliografia}

Documentación sobre el paquete \textbf{skak}: \\
\url{https://es.sharelatex.com/learn/Chess_notation} \\

Notación algebraica: \\
\url{https://es.wikipedia.org/wiki/Notaci%C3%B3n_algebraica} \\

Ejemplos: \\
\url{https://es.wikipedia.org/wiki/Inmortal_(partida_de_ajedrez)} \\
\url{https://es.wikipedia.org/wiki/Siempreviva_(partida_de_ajedrez)} \\
\url{http://www.ajedrezdeataque.com/12%20Visor/Visor1/Mackenzie-Mason.htm} \\
\end{document}
