\documentclass[11pt,a4paper]{article}

% Packages
\usepackage[utf8]{inputenc}
\usepackage[spanish, es-tabla]{babel}
\usepackage{caption}
\usepackage{listings}
\usepackage{adjustbox}
\usepackage{enumitem}
\usepackage{boldline}
\usepackage{amssymb, amsmath,amsthm}
\usepackage[margin=1in]{geometry}
\usepackage{xcolor}
\usepackage{soul}
\usepackage{upgreek}
\usepackage{float}

% Meta
\title{Entrega 3}
\author{José Antonio Álvarez}
\date{\today}

% Custom
\providecommand{\abs}[1]{\lvert#1\rvert}
\setlength\parindent{0pt}
% Redefinir letra griega épsilon.
\let\epsilon\upvarepsilon
% Fracciones grandes
\newcommand\ddfrac[2]{\frac{\displaystyle #1}{\displaystyle #2}}
% Primera derivada parcial: \pder[f]{x}
\newcommand{\pder}[2][]{\frac{\partial#1}{\partial#2}}

\begin{document}

\maketitle

\textbf{Ejercicio 13.} Encuentra una gramática libre de contexto en forma normal de Chomsky que genere el siguiente lenguaje:

$$ L = \{ucv : u, v \in \{0, 1\} \text{ y nº de subcadenas } 01 \text{ en } u \text{ es igual al nº de subcadenas } 10 \text{ en } v\} $$

Comprueba con el algoritmo \textbf{CYK} si la cadena \emph{010c101} pertenece al lenguaje generado por la gramática.

\begin {enumerate} 

\item Gramática libre del contexto que general el lenguaje:
	
$$ S \rightarrow S_1AS_1 | S_1BS_0 | S_0CS_1 | S_0DS_0 | YXY $$
$$ Y \rightarrow S_1 | S_0 | \epsilon $$
$$ S_0 \rightarrow 0 | 0S_0$$
$$ S_1 \rightarrow 1 | 1S_1$$
$$ A \rightarrow 1A | A1 | B0 | 0C | X | c $$
$$ B \rightarrow 1B | B0 | 0D | X | c $$
$$ C \rightarrow 0C | C1 | C0 | X | c $$
$$ D \rightarrow 0D | D0 | X | c $$
$$ X \rightarrow 01A10 $$

Donde $A$ siempre tendrá $1$ a ambos lados; $B$, $1$ a la izquierda y $0$ a la derecha; $C$, $0$ a la izquierda y $1$ a la derecha; y por último $D$, $0$ a ambos lados.

$S$ y sus transiciones son introducidos (en vez de empezar directamente en $A$) para que tanto $u$ como $v$ no sean vacías.

$Y$ ha sido incluida para aquellas palabras que del tipo $0001c10$, donde $v$ (lo mismo para $u$ con otras palabras) no está vacía pero no incluye ningún elemento además de la subcadena $10$. Estas palabras no están contempladas si no incluimos este estado.

\item Gramática en forma normal de Chomsky que genere el lenguaje:

$$ S \rightarrow S_1E_2 | S_0E_3 | YX | XY | S_0X_0$$
$$ E_2 \rightarrow AS_1 | BS_0 $$
$$ E_3 \rightarrow CS_1 | DS_0 $$
$$ S_0 \rightarrow 0 | S_0S_0$$
$$ S_1 \rightarrow 1 | S_1S_1$$
$$ A \rightarrow S_1A | AS_1 | BS_0 | S_0C | S_0X_0 | c $$
$$ B \rightarrow S_1B | BS_0 | S_0D | S_0X_0 | c $$
$$ C \rightarrow S_0C | CS_1 | DS_0 | S_0X_0 | c $$
$$ D \rightarrow S_0D | DS_0 | S_0X_0 | c $$
$$ X \rightarrow S_0X_0$$
$$ X_0 \rightarrow S_1X_1$$
$$ X_1 \rightarrow AX_2$$
$$ X_2 \rightarrow S_1S_0$$

\item Apliquemos ahora \textbf{CYK} para la cadena \emph{010c010}:

\begin{table}[h]
	\centering
	\caption{\textbf{CYK} - ejercicio 13}
	\label{my-label}
	\begin{tabular}{lllllll}
		0                          & 1                                  & 0                                                                                        & c                                                                                     & 0                         & 1                         & 0                         \\ \hline
		\multicolumn{1}{|l|}{S\_0} & \multicolumn{1}{l|}{S\_1}          & \multicolumn{1}{l|}{S\_0}                                                                & \multicolumn{1}{l|}{A, B, C, D}                                                       & \multicolumn{1}{l|}{S\_0} & \multicolumn{1}{l|}{S\_1} & \multicolumn{1}{l|}{S\_0} \\ \hline
		\multicolumn{1}{|l|}{}     & \multicolumn{1}{l|}{X\_2}          & \multicolumn{1}{l|}{A, B, C, D}                                                          & \multicolumn{1}{l|}{\begin{tabular}[c]{@{}l@{}}A, B, C, D, \\ E\_2,E\_3\end{tabular}} & \multicolumn{1}{l|}{}     & \multicolumn{1}{l|}{X\_2} &                           \\ \cline{1-6}
		\multicolumn{1}{|l|}{}     & \multicolumn{1}{l|}{A, B}          & \multicolumn{1}{l|}{\begin{tabular}[c]{@{}l@{}}S, A, B, C,\\ D, E\_2, E\_3\end{tabular}} & \multicolumn{1}{l|}{A, C, E\_, E\_3}                                                  & \multicolumn{1}{l|}{}     &                           &                           \\ \cline{1-5}
		\multicolumn{1}{|l|}{}     & \multicolumn{1}{l|}{S, A, B, E\_2} & \multicolumn{1}{l|}{\begin{tabular}[c]{@{}l@{}}S, A, C, E\_2,\\ E\_3\end{tabular}}       & \multicolumn{1}{l|}{X\_1}                                                             &                           &                           &                           \\ \cline{1-4}
		\multicolumn{1}{|l|}{}     & \multicolumn{1}{l|}{S, A}          & \multicolumn{1}{l|}{X\_1}                                                                &                                                                                       &                           &                           &                           \\ \cline{1-3}
		\multicolumn{1}{|l|}{}     & \multicolumn{1}{l|}{X\_0, X\_1}    &                                                                                          &                                                                                       &                           &                           &                           \\ \cline{1-2}
		\multicolumn{1}{|l|}{S}    &                                    &                                                                                          &                                                                                       &                           &                           &                           \\ \cline{1-1}
	\end{tabular}
\end{table}

\end{enumerate}

\textbf{Ejercicio 15.} Encuentra una gramática libre de contexto en forma normal de Chomsky que genere los
siguientes lenguaje definidos sobre el alfabeto $\{a, 0, 1\}$ :
$$L_1 = \{auava : u,v \in \{0, 1\}^+ \text { y } u^{-1} = v\} $$
$$L_2 = \{uvu : u,v \in \{0, 1\}^+ \text{ y } u^{-1} = v\} $$
Comprueba con el algoritmo \textbf{CYK} si la cadena \emph{a0a0a} pertenece a $L_1$ y la cadena \emph{011001} pertenece al lenguaje $L_2$ .

\begin{enumerate}
	\item Gramática libre del contexto que genera $L_1$:
		
	$$ S \rightarrow aAa $$
	$$ A \rightarrow XAX | a $$
	$$ X \rightarrow 0|1 $$
		
	\item Gramática en forma normal de Chomsky que genera $L_1$:
	
	$$ S \rightarrow AE_1 $$
	$$ E_1 \rightarrow BA $$
	$$ A \rightarrow a $$
	$$ B \rightarrow XE_2 | a $$
	$$ E_2 \rightarrow BX $$
	$$ X \rightarrow 0|1 $$
	
	\item Apliquemos ahora \textbf{CYK} para la cadena \emph{a0a0a}:
	
	\begin{table}[h]
		\centering
		\caption{\textbf{CYK} - ejercicio 15}
		\label{my-label}
		\begin{tabular}{lllll}
			a                          & 0                         & a                         & 0                      & a                         \\ \hline
			\multicolumn{1}{|l|}{A, B} & \multicolumn{1}{l|}{X}    & \multicolumn{1}{l|}{A, B} & \multicolumn{1}{l|}{X} & \multicolumn{1}{l|}{A, B} \\ \hline
			\multicolumn{1}{|l|}{E\_2} & \multicolumn{1}{l|}{}     & \multicolumn{1}{l|}{E\_2} & \multicolumn{1}{l|}{}  &                           \\ \cline{1-4}
			\multicolumn{1}{|l|}{}     & \multicolumn{1}{l|}{B}    & \multicolumn{1}{l|}{}     &                        &                           \\ \cline{1-3}
			\multicolumn{1}{|l|}{}     & \multicolumn{1}{l|}{E\_1} &                           &                        &                           \\ \cline{1-2}
			\multicolumn{1}{|l|}{S}    &                           &                           &                        &                           \\ \cline{1-1}
		\end{tabular}
	\end{table}
	
	\item $L_2$ no es libre del contexto. Lo demostraremos utilizando el lema de bombeo y la cadena $f = 0^n1^n1^n0^n0^n1^n$ (obviamente del lenguaje tomando $u = 0^n1^n$) Reescribamos el lenguaje para que la prueba sea más sencilla:
	
		$$L_2 = \{u_1vu_2 : u_1,u_2,v \in \{0, 1\}^+ \text{, } u_1^{-1} = v \text{ , } u_1 = u_2 \} $$
	
	Aplicando el lema de bombeo, sea una descomposición de $f$ de la forma $f = abcde$ donde $|bcd| \leq n$, $|b| \geq 1$ y $|d| \geq 1$. Llamemos $x = bcd$.
	
	\begin{enumerate}
		\item Si $x$ unicamente toma elementos de $u_1$, al bombear no se cumple ni que $v = u_1$ ni que $u_1 = u_2$.
		\item Si $x$ toma elementos tanto de $u_1$ como de $v$, entonces $x = 1^k, k \in \{1, ... , n\}$. Bombeando unicamente unos, $u_1 \neq u_2$.
		\item Si $x$ toma elementos tanto de $v$ como de $u_2$, entonces $x = 0^k, k \in \{1, ... , n\}$. Bombeando unicamente ceros,
		$u_1 \neq u_2$.
		\item Finalmente, si $x$ unicamente toma elementos de $u_2$, al bombear no se cumple ni que $v = u_1$ ni que $u_1 = u_2$.
	\end{enumerate}
	
	Por tanto el lenguaje no es libre del contexto. Es obvio entonces que no podemos encontrar una gramática en forma normal de Chomsky que lo genere. Como el algoritmo \textbf{CYK} unicamente se aplica a gramáticas de esta forma, no puedo aplicarlo. Aún así es obvio que la cadena \emph{011001} pertenece al lenguaje (de hecho es $f$ para $n = 1$).	 
\end{enumerate}


\end{document}